\chapter*{General Introduction}

\addcontentsline{toc}{chapter}{General Introduction}
\begin{spacing}{1.2}
%==================================================================================================%

The software engineering industry is experiencing a major shift driven by the rapid evolution of artificial intelligence and the growing demand for scalable, high-quality code development practices. As development teams grow and systems become more complex, ensuring consistent code quality and adherence to best practices presents an ongoing challenge, especially in large organizations managing massive codebases across distributed teams.\\

Traditional static analysis tools and linters, while helpful, often fall short when it comes to identifying nuanced or subjective coding issues that depend on context or internal guidelines. In fast-paced environments like those of YouTube engineering teams, developers need intelligent, responsive tools that go beyond rule-based checks to provide deeper insights and real-time guidance, without adding friction to their daily workflows.\\

This graduation project, conducted at Google Zurich, explores the integration of generative artificial intelligence into an internal Integrated Development Environment (IDE) to support software engineers in their day-to-day coding activities. It introduces a Large Language Model (LLM)-powered agent capable of performing in-depth code analysis, detecting complex violations, and offering clear, contextual suggestions for improvement. Embedded directly within the IDE through an extension, this intelligent assistant aims to enhance code quality, reduce technical debt, and support developer productivity at scale.\\

This report begins by presenting the context and challenges of applying AI in modern software development. It then outlines the design and integration of the AI agent, followed by an evaluation of its contribution to improving engineering workflows within a high-impact, real-world environment.




\end{spacing}


