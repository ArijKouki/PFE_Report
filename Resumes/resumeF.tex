\chapter*{Résumé}
\addcontentsline{toc}{chapter}{Résumé}
%===================================================================

Ce projet a été réalisé chez Google Zurich dans le cadre d’un Diplôme National d’Ingénieur en Génie Logiciel. Il explore l’intégration de l’intelligence artificielle générative dans le processus de développement logiciel en incorporant un agent basé sur un Large Language Model (LLM) au sein d’un Environnement de Développement Intégré (IDE) interne.

Cet agent effectue une analyse approfondie du code afin de détecter des violations complexes ou subjectives que les outils d’analyse statique traditionnels peuvent négliger. En générant des explications claires et compréhensibles ainsi que des suggestions concrètes, il aide les développeurs, notamment ceux contribuant à YouTube, à maintenir une haute qualité de code et à respecter les bonnes pratiques. Intégré de manière transparente au sein du flux de travail via une extension de l’IDE, l’agent améliore la productivité et contribue à la réduction de la dette technique sans perturber l’expérience de développement.

Ce travail met en évidence le potentiel des outils assistés par l’IA pour transformer l’expérience des développeurs et ouvre des perspectives pour l’avenir des environnements de développement intelligents.\\

\textbf{Mots-clés : Génie Logiciel, Intelligence Artificielle Générative, Intégration IDE, Qualité du Code, Productivité des Développeurs}