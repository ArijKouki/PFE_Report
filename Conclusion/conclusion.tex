\chapter{Conclusion and Perspectives}
%==============================================================================
\pagestyle{fancy}
\fancyhf{}
\fancyhead[R]{\bfseries\rightmark}
\fancyfoot[R]{\thepage}
\renewcommand{\headrulewidth}{0.5pt}
\renewcommand{\footrulewidth}{0pt}
\renewcommand{\chaptermark}[1]{\markboth{\MakeUppercase{\chaptername~\thechapter. #1 }}{}}
\renewcommand{\sectionmark}[1]{\markright{\thechapter.\thesection~ #1}}

\begin{spacing}{1.2}
%==============================================================================

This project has developed and implemented an intelligent assistance system integrated into YouTube's internal development environment, aimed at enforcing framework-specific best practices in real time. The system addresses a key challenge in large-scale software engineering: the absence of immediate, contextual feedback on internal frameworks. Unlike existing tools that focus on syntax or public libraries, our solution delivers intelligent, context-aware guidance directly within the developer workflow, reducing technical debt and improving code consistency.

The main contribution consists of an AI agent based on Large Language Models (LLMs) that orchestrates five specialized tools for code analysis, explanation, and fix generation, coupled with a YouTube IDE extension offering intuitive entry points and a two-tier diagnostic state management system that balances responsiveness with precision. Development leveraged Python for the agent, TypeScript and the VS Code API for the extension, and Google’s internal AI platform for LLM integration, following an agile Kanban-based workflow.

Design decisions throughout the project were guided by empirical testing and data-driven insights, ensuring that chosen approaches balanced performance, reliability, and usability. Significant technical challenges were addressed, including stale diagnostic handling through a two-tier system, concurrency control for stable performance, and robust semantic evaluation using an LLM-as-a-Judge methodology.

Looking ahead, the next step is implementing a Tiered Analysis Approach that combines a lightweight, rule-based linter for fast detection of simple issues with the LLM agent reserved for complex, subjective cases. This hybrid strategy promises faster, cheaper, and more effective feedback, further enhancing the developer experience. A teammate is already building the linter, and integration plans are in place.

Overall, this project demonstrates the feasibility and value of embedding AI agents into development environments for enforcing framework-specific best practices. It provides a strong foundation for future improvements and represents a meaningful contribution to advancing intelligent developer tools at YouTube.

%==============================================================================
\end{spacing}